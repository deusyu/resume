\documentclass{resume}

\newcommand{\en}[1]{#1}
\newcommand{\zh}[1]{}

\zh{\usepackage{xeCJK}}
\zh{\setCJKmainfont{Source Han Serif SC}}
\zh{\setCJKsansfont{Source Han Sans SC}}
\zh{\setCJKmonofont{Source Han Sans SC}}

\begin{document}

\name{\en{Jianxin Qiu}\zh{邱建鑫}}
\basicInfo{
      \email{superqjx@hotmail.com} \textperiodcentered\
      \github[imtsuki]{https://github.com/imtsuki}
}

\section{\en{Education}\zh{教育经历}}
\en{\datedsubsection{\textbf{Beijing University of Posts and Telecommunications}, Undergraduate}{09/2017 -- Present}}
\zh{\datedsubsection{\textbf{北京邮电大学}, 在读本科}{2017/09 -- 至今}}
\begin{itemize}
      \item \en{Major: Data Science and Big Data Technology, Ranked 1 out of 63, Anticipated Graduation Date: 06/2021}
            \zh{数据科学与大数据技术(计算机学院实验班), 排名 1/63, 2021 年毕业}
      \item GPA: 90.47/100
\end{itemize}

\section{\en{Work Experience}\zh{工作经历}}
\en{\datedsubsection{\textbf{\href{https://www.smartx.com/global/}{SmartX Inc.}}, Beijing, China}{09/2019 -- 01/2020}}
\zh{\datedsubsection{\textbf{\href{https://www.smartx.com/}{北京志凌海纳科技有限公司(SmartX Inc.)}}}{2019/09 -- 2020/01}}
\en{\role{Distributed Storage Systems}{Software Engineer Intern, C++}}
\zh{\role{分布式存储系统(ZBS)}{C++开发实习}}
\begin{itemize}
      \item \en{Improved the long task execution module, including backup storage parallelization, QoS and task status management.}
            \zh{改进了 ZBS 的长任务执行模块(Task Center),如支持备份存储过程批并行化、QoS 限速及任务的状态控制等}
      \item \en{Implemented Hadoop-like command line tools for the NFS interface of the storage service.}
            \zh{为存储服务的 NFS 接口实现了一整套类似于 Hadoop HDFS 的命令行工具}
      \item \en{Investigated and tuned the performance of MySQL running on ZBS at kernel level.}
            \zh{在内核层面调查并调优了 MySQL 运行在 ZBS 上的一些性能问题}
      \item \en{Learned a lot about distributed systems and storage.}
            \zh{学到了很多分布式和系统存储相关的知识}
\end{itemize}

\section{\en{Research Experience}\zh{研究经历}}
\en{\datedsubsection{\textbf{Network and Big Data Technology R\&D Center}, Tsinghua University}{02/2020 -- Present}}
\zh{\datedsubsection{\textbf{清华大学网络大数据技术研究中心}}{2020/02 -- 至今}}
\en{\role{RISC-V TEE}{Research Intern}}
\zh{\role{RISC-V 可信执行环境}{科研实习}}
\begin{itemize}
      \item \en{Implemented committed instruction flow collection based on RocketChip running on FireSim.}
            \zh{实现了 FireSim 上基于 RocketChip 的指令流收集}
\end{itemize}

\section{\en{Portfolios}\zh{个人项目}}
\datedsubsection{\textbf{xv7}}{\url{https://github.com/imtsuki/xv7}}
\en{Operating System implemented in Rust}
\zh{使用 Rust 编写的操作系统}
\begin{itemize}
      \item \en{``xv7'' means it's inspired by MIT6.828 and xv6.}
            \zh{项目目标为使用 Rust 重新实现 MIT6.828 和 xv6}
      \item \en{Implemented UEFI Bootloader, interrupt management and memory management.}
            \zh{实现了 UEFI Bootloader 、中断处理与内存管理}
      \item \en{Achieved memory safety in kernel with the help of Rust's safe abstractions and lifetimes.}
            \zh{借助 Rust 的抽象能力与生命周期概念实现内核中的内存安全}
      \item \en{Made contributions to \texttt{\href{https://github.com/rust-osdev}{rust-osdev}}, an organization aiming at providing tools useful for OS development in Rust.}
            \zh{为 Rust 操作系统开源组织 \texttt{\href{https://github.com/rust-osdev}{rust-osdev}} 贡献代码}
\end{itemize}

\en{\datedsubsection{\textbf{Raft Algorithm}}{08/2019}}
\zh{\datedsubsection{\textbf{分布式系统一致性算法 Raft}}{2019/08}}
MIT 6.824 Labs
\begin{itemize}
      \item \en{Implemented fault-tolerant Raft algorithm and K-V store based on that.}
            \zh{理解了 Raft 算法的原理,实现了容错的 Raft 算法,并在其基础上实现了 K-V 存取}
\end{itemize}

\section{\en{Personal Experience}\zh{个人经历}}
\en{\datedsubsection{\textbf{Cambridge Academic Development Seminar}, U.K.}{07/2018 -- 08/2018}}
\zh{\datedsubsection{\textbf{英国剑桥大学人工智能暑期交流项目}}{2018/07 -- 2018/08}}
\en{\role{Machine Learning}{Summer Program}}
\zh{\role{学生}{暑期交流}}
\begin{itemize}
      \item \en{Learned machine learning algorithms.}
            \zh{学习了经典机器学习算法}
      \item \en{Collaborated with others researching in machine learning applications and concerns.}
            \zh{在小组中担任领导角色,负责了小组展示的选题与组织}
\end{itemize}

\section{\en{Skills}\zh{技能}}
\begin{itemize}[parsep=0.25ex]
      \item \en{\textbf{Programming Languages}:
                  not limited to any specific language,
                  and experienced in Rust/C/C++,
                  comfortable with Python/C\#/Java/TypeScript (in random order).}
            \zh{\textbf{编程语言}:
                  熟悉各种编程语言,尤其熟悉 Rust/C/C++ 等,
                  了解 Python/C\#/Java/TypeScript 等(不分先后)}

      \item \en{\textbf{System}:
                  familiar with operating system concepts and design,
                  have experience in optimizing performance on kernel level using tools
                  like \texttt{strace} and \texttt{blktrace}.}
            \zh{\textbf{系统}:
                  熟悉各种操作系统内核的概念与设计,熟悉各种内核性能调优工具,例如 \texttt{strace} 和 \texttt{blktrace}}

      \item \en{\textbf{Distributed Systems}:
                  taken course MIT 6.824,
                  understand consensus algorithms like Raft and ZooKeeper,
                  have experience in distributed system development.}
            \zh{\textbf{分布式系统}:
                  熟悉 ZooKeeper 与 Raft 算法,有分布式系统开发经验}

      \item \en{\textbf{Machine Learning}:
                  familiar with general knowledge of machine learning.}
            \zh{\textbf{机器学习}:
                  熟悉经典机器学习算法}

      \item \en{\textbf{Developing Tools}:
                  experienced in Linux-based programming,
                  have experience with team tools like Jira, Git, etc.}
            \zh{\textbf{开发工具}:
                  十分熟悉 Linux,有 Jira、Git 等团队合作工具的经验}
\end{itemize}

\section{\en{Miscellaneous}\zh{其他}}
\begin{itemize}[parsep=0.25ex]
      \item \en{Interests: Distributed Systems and storage, databases and cloud applications.}
            \zh{兴趣:分布式系统、存储、数据库、云计算应用等}
      \item \en{Open-source Contributions: contributed to \texttt{@rust-analyzer, @rust-osdev, @jupyter, @pingcap}, etc.}
            \zh{开源贡献: 为 \texttt{@rust-analyzer, @rust-osdev, @jupyter, @pingcap} 等组织贡献过代码}
      \item \en{Selected Courses: OS (91), Network (96), Database (92), Algorithm (93).}
            \zh{主修课程:操作系统(91)、计算机网络(96)、数据库系统原理(92)、算法设计与分析(93)}
      \item \en{Language Level: English GRE V158 + Q170 + AW3.5.}
            \zh{语言水平:英语 GRE V158 + Q170 + AW3.5}
      \item \textit{Meritorious Winner}, Mathematical Contest In Modeling 2019
\end{itemize}

\end{document}

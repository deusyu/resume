\documentclass{resume}

\newcommand{\en}[1]{#1}
\newcommand{\zh}[1]{}

\zh{\usepackage{xeCJK}}
\zh{\setCJKmainfont{SourceHanSerifSC-Medium}}
\zh{\setCJKsansfont{SourceHanSerifSC-Medium}}
\zh{\setCJKmonofont{SourceHanSerifSC-Medium}}

\begin{document}

\name{\en{Huarui Chen}\zh{陈华睿}}
\basicInfo{
      \email{chenhr1130@163.com} \textperiodcentered\
      \phone{+(86) 152-8989-2071} \textperiodcentered\
      \github[hi-Ernest]{https://github.com/hi-Ernest}
      \homepage[Blog]{https://hi-ernest.github.io/}
}

\section{\en{Education}\zh{教育经历}}
\en{\datedsubsection{\textbf{Chongqing University of Technology}, Bachelor's Degree}{09/2016 -- 06/2020}}
\zh{\datedsubsection{\textbf{重庆理工大学}, 本科}{2016/09 -- 2020/06}}
\begin{itemize}
      \item \en{Major: Software Engineering}
            \zh{软件工程}
      \item \en{Key Courses: OS, Network,  Database,Algorithm, Compiler Principle.}
            \zh{主修课程:操作系统、计算机网络、数据库系统原理、算法设计与分析、编译原理}
\end{itemize}

\section{\en{Work Experience}\zh{工作经历}}
\en{\datedsubsection{\textbf{\href{https://www.bytedance.com/}{ByteDance Inc.}}, Beijing, China}{04/2021 -- now}}
\zh{\datedsubsection{\textbf{\href{https://www.bytedance.com/}{字节跳动科技有限公司,北京}}}{2021/04 -- 至今}}
\en{\role{Supply Chain and Logistics}{Backend Development Engineer}}
\zh{\role{供应链与物流}{后端开发工程师}}
\begin{itemize}
      \item \en{Participate in the design of an abstract monitoring model, receive messages from different upstream data sources, convert them into configured events as triggers to start/end monitoring, find the corresponding timeout rules through events, and generate monitoring orders for the corresponding rules to perform monitoring. The monitoring list needs to be filled with data information, and the RPC interface needs to be called multiple times. In order to reduce multiple repeated requests, the result of building a memory cache request is stored in the context context; for infrequently changing rules and events, Timer is used to periodically obtain the cache from ByteSQL and store it in memory. , reduce query operations on DB.}
            \zh{参与设计抽象的监控模型,接收上游不同数据源的消息,转化为配置的事件作为触发开始/完结监控,通过事件找到对应的超时规则,生成对应规则的监控单进行履行监控。监控单需要填充数据信息,需要调用多次RPC接口,为了减少多次重复请求,构建内存缓存器请求结果存在context上下文中;针对不频繁变更规则和事件,采用Timer定时从ByteSQL获取缓存到内存当中,减少对DB查询操作。}
\end{itemize}

\en{\datedsubsection{\textbf{\href{https://www.hupu.com/}{Hupu Inc.}}, Shanghai, China}{07/2020 -- 01/2021}}
\zh{\datedsubsection{\textbf{\href{https://www.hupu.com/}{虎扑(上海)文化传播公司,上海}}}{2020/07 -- 2021/01}}
\en{\role{Product Development Group}{Backend Development Engineer}}
\zh{\role{产品研发部}{后端开发工程师}}
\begin{itemize}
            \item \en{Participate in the reconstruction of the basketball schedule list, responsible for the transformation of the schedule data, as well as the design and implementation of the schedule list API connection.}
            \zh{参与重构篮球赛程列表,负责赛程数据的转化,以及赛程列表API接又设计与实现。}
            \item \en{Reduce the amount of interface access through cache (local cache, Redis cache); use thread pool to synchronize asynchronously and synchronize game data in real time.}
            \zh{通过缓存减少接口访问量(本地缓存、Redis缓存);使用线程池并发异步,实时同步比赛数据 }
\end{itemize}

\section{\en{Skills}\zh{技能}}
\begin{itemize}[parsep=0.25ex]
       \item \en{\textbf{Programming Languages}:
                        \textbf{multilingual} (not limited to any specific language), 
                  	experienced in Golang/Java,
                  	comfortable with Rust/Python/C/C++.}
            \zh{\textbf{编程语言}:
                 \textbf{泛语言}(编程不受特定语言限制),
                  熟悉 Golang/Java,
               了解 Rust/Python/C/C++ 等}
                           
       \item \en{\textbf{Data Structures and Algorithms}:
                        familiar with common data structures (Array, List, Stack, Queue, Map, Set, BinTree, BST), 
                        understand AVL, RBtree, B/B+ tree, skip list, familiar with commonly used sorting algorithms (Bubble, Insert, Select, Merge, QuickSort, HeapSort, Bucket, Count).}
                \zh{\textbf{数据结构与算法}:
                		{熟悉常用数据结构}(Array、List、Stack、Queue、Map、Set、BinTree、BST),
		       了解AVL、RBtree、B/B+树、跳表;
                          熟悉常用的排序算法(冒泡、插入、选择、归并、快排、堆排、桶、计数)。}

      \item \en{\textbf{Distributed System/Database}:
                  Experience in tuning and deployment of TiDB, know the basic use of K8s and tidb-operator.
                  taken course MIT 6.824 and PingCAP's Talent Plan,
                  understand the basic theory of distributed system/database,
                  including but not limited to algorithms such as Raft and Percolator.}
              \zh{\textbf{分布式系统/数据库}:
                  有分布式数据库 TiDB 的调优开发以及部署经验,了解 K8s 以及 tidb-operator 的基本使用。
                  自主学习了 MIT 6.824 和 PingCAP's Talent Plan 等课程,
                  了解分布式系统/数据库的基本理论,包括但不限于 Raft 和 Percolator 等算法}
                  
      \item \en{\textbf{Developing Tool}:
                  familiar with Linux-based programming,
                  have experience with team tools like Git, Jira, etc.}
            \zh{\textbf{开发工具}:
                  熟悉 Linux,有 Git、Jira 等团队协作工具的使用经验}
   
     \item \en{\textbf{Others}:
                  have experience using MySQL/Redis/RocketMQ/Elasticsearch, 
                  understand Docker and Docker orchestration concepts, 
                  have experience in developed logistics performance monitoring to support multi-dimensional query, 
                  used Elasticsearch to solve multi-condition query problems.}
          \zh{\textbf{其它}:	 	 
                  有 MySQL/Redis/RocketMQ/Elasticsearch 使用经验,了解容器及容器编排相关概念,
               开发过物流履约监控支持多维度查询,使用 Elasticsearch 解决多条件查询问题。}
\end{itemize}

\section{\en{Miscellaneous}\zh{其他}}
\begin{itemize}
      \item \en{Interests: computer systems and architecture, parallel computing, databases and cloud applications.}
            \zh{兴趣:分布式系统、存储、数据库、云计算应用等。}
      \item \en{Open-source Contributions: contributed to \texttt{@rust-analyzer, @rust-osdev, @jupyter, @pingcap}, etc.}
            \zh{开源贡献: 为 \texttt{@rust-analyzer, @rust-osdev, @jupyter, @pingcap} 等组织贡献过代码。}
      \item \en{\textbf{Passionate about sharing}: I have been writing on my personal blog for years and have accumulated several high-quality technical sharing articles.}
            \zh{\textbf{热爱分享}: 在个人博客上常年坚持写作,积累高质量技术分享文章数篇}
       \item \en{\textbf{Language Level}: Ability to conduct daily conversation and essay reading, experience in English presentation.}
            \zh{\textbf{语言水平}:能够进行日常对话和论文阅读,有英文演讲经验}
       \item \en{\textbf{Personal tags}: self-driven, quick learner, earnest curiosity, open source lover.}
            \zh{\textbf{个人标签}:自驱动,学习能力强,做事认真,保持好奇,热爱开源}
\end{itemize}

\end{document}

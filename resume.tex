\documentclass{resume}

\newcommand{\en}[1]{#1}
\newcommand{\zh}[1]{}

\zh{\usepackage{xeCJK}}
\zh{\setCJKmainfont{SourceHanSerifSC-Medium}}
\zh{\setCJKsansfont{SourceHanSerifSC-Medium}}
\zh{\setCJKmonofont{SourceHanSerifSC-Medium}}


\begin{document}

\name{\en{Dongyu Cao}\zh{曹东昱}}
\basicInfo{
      \email{dongyuchiao@gmail.com} \textperiodcentered\
      \phone{+(86) 1800-1212-101} \textperiodcentered\
      \wechat{cdeusyu}\textperiodcentered\
      \homepage[blog]{https://deusyu.app/}
}

\section{\en{Education}\zh{教育经历}}
\en{\datedsubsection{\textbf{Chongqing University of Technology}, Master's Degree}{09/2020 -- 06/2023}}
\zh{\datedsubsection{\textbf{重庆理工大学}, 硕士}{2020/09 -- 2023/09}}
\begin{itemize}
      \item \en{Major: Computer Technology}
            \zh{计算机技术}
      \item \en{Focused on NLP (Text2SQL) research and published papers in EI-indexed journals.}
            \zh{主要做NLP(Text2SQL)相关研究,发表EI论文}
\end{itemize}
\en{\datedsubsection{\textbf{Chongqing University of Technology}, Bachelor's Degree}{09/2016 -- 06/2020}}
\zh{\datedsubsection{\textbf{重庆理工大学}, 本科}{2016/09 -- 2020/06}}
\begin{itemize}
      \item \en{Major: Software Engineering}
            \zh{软件工程}
\end{itemize}

\section{\en{Work Experience}\zh{工作经历}}
\en{\datedsubsection{\textbf{\href{https://www.kuaishou.com/en/}{Kwai Inc.}}, Beijing, China}{03/2022 -- now}}
\zh{\datedsubsection{\textbf{\href{https://www.kuaishou.com/about/}{快手,北京}}}{2022/03 -- 至今}}
\en{\rolewithdate{Resource Management Platform}{Backend Development Engineer}{07/2023 -- now}}
\zh{\rolewithdate{资源管理平台}{后端开发工程师}{2023/07 -- 至今}}
\begin{itemize}
      \item \en{Responsible for the development of the CMP multi-cloud management platform. Participated in the design and implementation of the public cloud testing process, Huawei Cloud's access, Alibaba Cloud, Tencent Cloud LB related product access and development, and also involved in platform operation and maintenance, ensuring user public cloud experience optimization and efficient platform resource management.}
            \zh{负责CMP多云管理平台的研发,期间参与公有云测试流程的设计与实现,华为云的接入,阿里云、腾讯云LB相关产品的接入与研发,同时参与平台运维,确保用户公有云体验优化与平台资源管理高效。}
            \en{\\Tech Stack Used: Django REST framework, Go, Python, Redis, MySQL\\}
            \zh{\\使用的技术栈: Django REST framework, Go, Python, Redis, MySQL\\}
            \en{Tech Stack for Refactoring: ksboot, Redis, kconf, xxl-job, Mybatis-Plus}
            \zh{重构使用的技术栈: ksboot, Redis, kconf, xxl-job, Mybatis-Plus}
            \begin{enumerate}
                  \item \en{Solely responsible for the development of the "Public Cloud Testing Process", perfected the setup of the ksboot new project from scratch, including but not limited to meeting modern software engineering design standards, unified exception handling, and RESTful style returns. Additionally, ensured the compatibility of the new system with nacos/star ring; successfully implemented the process call integration with the old cloud platform; switched scheduled tasks to the company's task scheduling platform; adopted a combination of frontend and backend for permission control, improved system security, and perfected the test process reconciliation function. Based on user feedback, I continuously fixed defects and upgraded system functions, significantly improving the efficiency of the test process and the maintainability of the system. By fully managing public cloud testing resources, I implemented process control, overdue reminders, and resource opening functions, greatly simplifying the operation and management of resources for different role users (such as ordinary users and SREs), and improving the convenience and efficiency of the platform.}
                        \zh{独自负责了「公有云测试流程」的开发,完善了ksboot新项目从零到一的搭建,包括不限于符合现代软件工程的设计规范、统一异常处理及RESTful风格返回。
                              此外,确保了新系统与nacos/星环的兼容性;并顺利实现了与旧云集平台的流程调用集成;定时任务切换为公司的任务调度平台;
                              权限采用前后端结合来控制,提高了系统安全性,并完善了测试流程对账相关的功能,基于用户反馈,我持续修复缺陷并升级系统功能,显著提升了测试流程的效率和系统的可维护性。
                              通过全面管理公有云测试资源,我实现了流程控制、逾期提醒及资源开通等功能,极大地简化了不同角色用户(如普通用户和SRE)对资源的操作和管理,提升了平台的使用便利性和效率。}
                  \item \en{As a core developer in the full process access of Huawei Cloud products to CMP, I was responsible for the design and implementation of network security group, PaaS resource management (especially MongoDB), and cross-language (Django, SpringBoot, Go) development work, significantly improving the efficiency of cloud resource management and system security.}
                        \zh{在华为云产品全流程接入CMP中担任核心开发者,负责设计与实施网络安全组、PaaS资源管理(特别是MongoDB)以及跨语言(Django、SpringBoot、Go)开发工作,显著提高了云资源的管理效率和系统安全性。}
                  \item \en{Under the multi-cloud environment, I led the design and implementation of the load balancing function upgrade, especially:}
                        \zh{在多云环境下,主导了负载均衡功能升级的设计与实现,特别是:}
                        \begin{itemize}
                              \item \en{Designed and implemented the access of Alibaba Cloud Network Load Balancer (NLB), optimized the cloud service SDK adaptation process, and ensured the efficient integration of products from different cloud manufacturers.}
                                    \zh{设计并实现了阿里云网络型负载均衡NLB的接入,优化了云服务SDK的适配流程,确保了不同云厂商产品的高效整合。}
                              \item \en{Responsible for the function expansion and version upgrade of Tencent Cloud Load Balancer (LB), introduced shared and performance capacity specifications, significantly improving the flexibility and performance of the service.}
                                    \zh{负责腾讯云负载均衡LB的功能扩展与版本升级,引入共享型和性能容量型规格,显著提升了服务的灵活性和性能。}
                        \end{itemize}
                  \item \en{In the Huawei Cloud access project, I led the use of the Java tech stack to refactor the original 1 Python and 3 Go services, successfully integrating them into 2 efficient Java applications. From being dispersed to integrated, this transformation significantly reduced system maintenance costs and significantly improved system stability.}
                        \zh{在华为云接入项目中,我主导使用Java技术栈对原有1个Python和3个Go服务的重构,成功整合为2个高效的Java应用。由分散到整合,这一改造显著降低了系统维护成本,并显著提升了系统稳定性。\\}
            \end{enumerate}
\end{itemize}
\en{\rolewithdate{Resource Management Platform}{Backend Development Intern}{03/2022 -- 06/2023}}
\zh{\rolewithdate{资源管理平台}{后端开发实习生}{2022/03 -- 2023/06}}
\begin{itemize}
      \item \en{Responsible for the development of the reconciliation platform, built the first phase of the reconciliation platform from scratch, including supplier bill upload and download, parsing and mapping, cloud merchant bill access, supplier alarm function, external email module design and implementation, bill report implementation, completed the reconciliation of self-built CDN, PCDN, improved efficiency (the billing cycle was reduced from the original 25 man-days (5 people5 days) to 5 man-days (2.5 people2 days), improved efficiency by 400\%, the reconciliation payment work was upgraded from the original 15 man-days to 7 man-days, improved efficiency by 114\%)}
            \zh{负责对账平台的研发工作,从零到一搭建了对账平台1期,包括供应商账单的上传与下载,解析与映射,云商账单的接入,供应商告警功能,对外邮件模块的设计与实现,账单报表的实现,完成了自建CDN、PCDN的对账,提升人效(结账周期由原来的25人天(5人5天),缩减至5人天(2.5人2天),提效400\%,对账付款工作由原来的15人天,提升至7人天,提效114\%)}
            \en{\\Tech Stack Used: Springboot, MySQL, Clickhouse, Redis, Easyexcel, Grafana}
            \zh{\\使用的技术栈: springboot, mysql, clickhouse, redis, easyexcel, grafana}
            \begin{enumerate}
                  \item \en{Utilized ClickHouse to achieve efficient storage and processing of supplier bills, significantly improved system processing speed through its excellent data insertion and query performance, combined with data rule aggregation to MySQL, achieved instant response and seamless experience for users when uploading bills, greatly optimized user operation process and satisfaction.}
                        \zh{利用ClickHouse实现供应商账单的高效存储与处理,通过其卓越的数据插入和查询性能,配合数据规则聚合至MySQL,显著提高了系统处理速度,实现了用户在上传账单时的即时响应和无感知体验,极大优化了用户操作流程和满意度}
                  \item \en{Designed email notification module to remind external suppliers to upload bills, bill validation reminder function, adjusted email sending rate and receipt, ensured suppliers upload and update bills on time and with quality, shortened the reconciliation cycle.}
                        \zh{设计邮件通知模块来提醒外部供应商上传账单,账单校验提醒功能,调整邮件发送速率与回执,确保供应商按时按质上传与更新账单,缩短了对账周期}
                  \item \en{Used the strategy pattern to optimize the bill parsing process, achieved flexible handling of different supplier bill formats, greatly improved development efficiency and system scalability, simplified the maintenance work of subsequent bill format changes.}
                        \zh{采用策略模式优化账单解析过程,实现了对不同供应商账单格式的灵活处理,极大提升了开发效率和系统的可扩展性,简化了后续账单格式变更的维护工作}
            \end{enumerate}
\end{itemize}
\begin{itemize}
      \item \en{Resource operation and asset management direction, supply chain + asset management (including RMS+CMP+power management) completed the full lifecycle management of company assets, budget + forecast + KAS bill as platform support to run a set of operation rules, based on the understanding of the resource management platform, expanded the existing business according to new requirements, added new function modules.}
            \zh{资源运营和资产管理方向,供应链+资产管理(包括RMS+CMP+电力管理)完成了公司资产的全生命周期管理,预算+预测+KAS账单作为平台支撑运行一套运营规则,在理解资源管理平台的基础上,根据新的需求在现有的业务上进行扩展,增添新的功能模块}
            \en{\\Tech Stack Used: SpringBoot, MyBatis, MySQL, Redis, Kafka, XxlJob, Kubernetes, Grafana}
            \zh{\\使用的技术栈: SpringBoot, MyBatis, MySQL, Redis, Kafka, XxlJob, Kubernetes, Grafana}
            \begin{enumerate}
                  \item \en{In the 2023 budget management project, responsible for key research and development work of the basic module and platform dashboard module. My contribution supported efficient budget collection and compilation processes, significantly improved the accuracy and efficiency of company financial planning, and shortened the budget compilation cycle by 20\%.}
                        \zh{在2023年预算管理项目中,负责基础模块和平台大盘模块的关键研发工作。我的贡献支持了高效的预算收集与编制流程,显著提高公司财务规划的精确度和效率,缩短预算编制周期20\%。}
                  \item  \en{In the execution of the 2023 budget, I was responsible for the development and testing of key functions such as budget transfer, budget addition, and budget month-end. The successful implementation of these functions greatly improved the flexibility and accuracy of budget execution, reduced the time required for budget adjustments, and improved the efficiency of budget tracking.}
                        \zh{在2023年度预算执行中,我负责开发和测试预算转移、预算追加及预算月结等关键功能。这些功能的成功实施极大提高了预算执行的灵活性和精确度,减少了预算调整所需的时间,提升了预算追踪的效率}
                  \item \en{Designed and implemented the supply chain arrival dashboard, optimized delivery tracking, significantly improved resource delivery efficiency.}
                        \zh{设计实现供应链到货大盘,优化交付追踪,显著提升资源交付效率}
                  \item \en{Participated in the asset management project using python automation scripts to improve the accuracy of resource (such as network card) management and strengthen online and offline synchronization.}
                        \zh{参与资产治理项目中使用python自动化脚本提升资源(如网卡)管理准确性,强化线上线下同步}\\
            \end{enumerate}

\end{itemize}

\en{\datedsubsection{\textbf{\href{https://www.dell.com/en-us/blog/tags/dell-emc/}{Dell Inc.}}, Shanghai, China}{10/2021 -- 02/2022}}
\zh{\datedsubsection{\textbf{\href{https://www.dell.com/en-us/blog/tags/dell-emc/}{DELL EMC,上海}}}{2021/10 -- 2022/02}}
\en{\role{VxRail VCF\&Netword Team}{Backend Development Intern}}
\zh{\role{VxRail VCF\&Netword Team}{后端开发实习生}}
\begin{itemize}
      \item \en{Participated in the iteration of the VxRail sub-platform VCF, responsible for the design and implementation of platform auto-configuration functionality and configuration customization GUI.}
            \zh{参与旗下产品VxRail子平台VCF的迭代,负责平台自动配置功能及配置定制化GUI的设计与实现}
            \begin{enumerate}
                  \item \en{Familiar with the overall architecture of the VxRail product, familiar with concepts such as HCI, SDS, SDDC; used vSAN to manage server clusters, wrote scripts to deploy configurations; effectively troubleshooted and quickly fixed bugs, ensuring the stable release of the product.}
                        \zh{熟悉VxRail产品的整体架构,熟悉HCI、SDS、SDDC等概念;使用vSAN管理服务器集群,编写脚本部署配置;有效排查并快速修复bug,保证了产品的稳定上线。}
                  \item \en{Adopted SpringBoot and Vue to achieve front-end and back-end separation, solved cross-domain and parameter passing issues, improved development efficiency and system maintainability.}
                        \zh{采用SpringBoot和Vue实现前后端分离,解决跨域和参数传递问题,提高了开发效率和系统的可维护性。}
                  \item \en{Developed a tree node construction and front-end rendering function based on JsonObject, used Tree widget to optimize the user interface, achieved efficient json data positioning and editing.}
                        \zh{开发了基于JsonObject的树节点构建和前端渲染功能,使用Tree控件优化用户界面,实现了高效的json数据定位与编辑。}
                  \item \en{Implemented json field filtering and multiple field mapping functions based on project configuration, enhanced the flexibility and accuracy of data processing.}
                        \zh{实现了基于项目配置的json字段过滤和多字段映射功能,增强了数据处理的灵活性和准确性。}
            \end{enumerate}
\end{itemize}

\section{\en{Skills}\zh{技能}}
\begin{itemize}[parsep=0.25ex]
      \item \en{Familiar with the Java language, understand the concept of object-oriented, master Java encapsulation, inheritance, and polymorphism features}
            \zh{熟悉Java语言,理解面向对象思想,掌握 Java 封装、继承和多态特性}

      \item \en{Understand Go, Python, maintain CMP and carry out functional research and development, understand object-oriented thinking, master the characteristics of encapsulation, composition and interface}
            \zh{了解Go、Python,维护CMP并进行了功能研发,理解面向对象思想,掌握封装、组合和接口的特性}

      \item \en{Familiar with common data structures (Array, List, Stack, Queue, Map, Set, Binary Tree, BST),
                  understand AVL Tree, Skip List, B/B+ Tree, Red-Black Tree}
            \zh{熟悉常见集合(List、Set、Map)、多线程并发(读写锁、阻塞队列、线程池、ConcurrentHashMap等),了解悲观、乐观并发控制以及MVCC等并发控制机制}

      \item \en{Familiar with common data structures (Array, List, Stack, Queue, Map, Set, Binary Tree, BST),
                  understand AVL Tree, Skip List, B/B+ Tree, Red-Black Tree}
            \zh{熟悉常用数据结构(Array、List、Stack、Queue、Map、Set、Binary Tree、BST),
                  了解AVL树、跳表、B/B+树、Red-Black Tree}

      \item \en{Familiar with mainstream frameworks such as SpringBoot, can quickly build and complete engineering configuration, such as Git, Maven, log optimization, configuration file, integration of hot deployment}
            \zh{熟悉SpringBoot等主流框架,可以快速构建并完成工程配置,如Git,Maven,日志优化,配置文件,集成热部署}

      \item \en{Understand JVM related knowledge, such as garbage collection (GC algorithm, garbage collector, Tri-color Marking) and class loading mechanism, parent delegation model}
            \zh{了解JVM相关知识,如垃圾回收(GC算法、垃圾收集器、三色标记)和类加载机制、双亲委派机制}

      \item \en{Proficient in using MVC model for project design, familiar with common design patterns such as singleton, strategy, and factory}
            \zh{熟练使用 MVC 模式进行项目设计,熟悉单例,策略,工厂模式等常见的设计模式}

      \item \en{Understand basic Git operations and Linux related basic knowledge, proficient in using Docker and other container technologies for CI/CD}
            \zh{了解Git基本操作以及Linux相关基本知识,熟练使用Docker等容器技术进行CI/CD}

      \item \en{Familiar with MySQL index principle, query performance optimization, understand storage engine (InnoDB, MyISAM)}
            \zh{熟悉 MySQL 索引原理、查询性能优化,了解存储引擎 (InnoDB、MyISAM)}

      \item \en{Familiar with database transaction characteristics, transaction isolation level, understand MVCC, Next-Key Locks, snapshot read and current read}
            \zh{熟悉数据库事务特性、事务隔离级别,了解 MVCC、Next-Key Locks、快照读与当前读}

      \item \en{Familiar with Redis data types, understand expired key deletion, data elimination strategy, RDB and AOF persistence mechanism}
            \zh{熟悉 Redis 数据类型,了解过期键删除、数据淘汰策略、RDB 和 AOF 持久化机制}

      \item \en{Familiar with TCP three-way handshake and four-way wave, understand TCP flow control and congestion control, understand the working principle of TLS's security mechanism}
            \zh{熟悉 TCP 三次握手和四次挥手,了解 TCP 流量控制和拥塞控制,了解 TLS的安全机制的工作原理}

      \item \en{Actively exploring LLM tools such as ChatGPT 4 and GitHub Copilot to improve coding efficiency. Continual learning, keep up with the technological frontier through technical blogs and open source projects, stimulate passion for innovation, adapt to technological changes.}
            \zh{积极探索如ChatGPT 4和GitHub Copilot等LLM工具,提升编码效率。持续学习,通过技术博客、开源项目保持技术前沿,激发创新热情,适应技术变革。}
\end{itemize}

\section{\en{Thanks}\zh{致谢}}
\begin{itemize}
      \item \en{Thank you for taking the time to review my resume. I look forward to the opportunity to work with you.}
            \zh{感谢您花时间阅读我的简历,期待能有机会和您共事}
\end{itemize}

\end{document}